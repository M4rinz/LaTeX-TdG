		% -------------------------- HEAD --------------------------
\documentclass[a4paper,11pt]{article}
\usepackage[T1]{fontenc}				% codifica dei font
\usepackage[utf8]{inputenc}				% lettere accentate da tastiera
\usepackage[italian]{babel}				% lingua del documento
%\usepackage{lipsum}						% per generare testo fittizio (placeholder text)

\usepackage{amssymb}
%\usepackage{amsmath}
\usepackage{mathtools}					% amsmath sotto steroidi
\usepackage{amsthm}						% Per creare teoremi e roba

\usepackage{mathdots}					% Migliora il comportamento dei tre puntini, estendendone le funzionalità
\usepackage{mathrsfs}					% Per dei caratteri matematici migliori: \mathscr{} e \mathcal{}
\usepackage{physics}					% Molte cose utili. La documentazione è molto ben fatta
%\usepackage{braket} 					% Per il comando \Set, e altre (poche) cose. Non so se serve sempre, col pacchetto physics...
\usepackage[italian]{varioref}			% Per usare il comando \vref{label}, che dà dei collegamenti più dettagliati
%\usepackage[notref]{showkeys}			% Mostra al lato del numero di etichetta il suo "nome"
%\usepackage[all,cmtip]{xy}				% xymatrix. Per fare diagrammi
\usepackage{microtype}					% Migliora la tipografia, permettendo ad alcuni elementi di sporgere leggermente

\usepackage{relsize}					% Per usare \mathbigger{} ecc
\usepackage{array}						% per fare gli array. Tabelle con matematica dentro
\usepackage{multirow}					% per usare il comando multirow
%\usepackage{faktor}						% Per i simboli dell'insieme quoziente

%\usepackage{listings}					% Per scrivere codice in LaTeX...
\usepackage{xcolor}						% ... e per colorarlo


\usepackage{graphicx}

\usepackage{hyperref}					% Migliora i collegamenti e links. Va caricato per ultimo


%\usepackage{tabularx}					% per fare tabelle. Carica il pacchetto array, per gli array.
%\usepackage{booktabs}					% i prossimi due pacchetti servono per le tabelle
%\usepackage{caption}
%\usepackage{subcaption}

%\usepackage{sidecap}					% Per inserire immagini e tabelle con didascalie a lato


	% TEOREMI CUSTOM:
\theoremstyle{definition}				% Definisce ambienti per Teoremi, esercizi, corollari... Con lo stile adeguato
\newtheorem{definizione}{Definizione}%[section]
\newtheorem{proposizione}{Proposizione}
\newtheorem*{proposizione*}{Proposizione}
\newtheorem{teorema}{Teorema}

\theoremstyle{remark}
\newtheorem{oss}{Osservazione}%[section]
\newtheorem*{oss*}{Osservazione}
\newtheorem*{notazione}{Notazione}

\theoremstyle{plain}
\newtheorem{esercizio}{Esercizio}%[section]
\newtheorem*{esercizio*}{Esercizio}
\newtheorem{corollario}{Corollario}[esercizio]
\newtheorem{lemma}{Lemma}%[esercizio]


 	% COMANDI CUSTOM: 	
\newcommand{\pf}{\mathcal{P}_{\text{\tiny F}}}					% Comando per le parti finite
\newcommand{\diffsim}{\mathbin{\scriptstyle\bigtriangleup}}		% Comando per la differenza simmetrica
\newcommand\restr[2]{{											% Commando for restriction. We make the whole thing an ordinary symbol
		\left.\kern-\nulldelimiterspace % automatically resize the bar with \right
		#1 % the function
		\vphantom{\big|} % pretend it's a little taller at normal size
		\right|_{#2} % this is the delimiter
}}
\newcommand{\comp}[1]{{#1}^{\complement}}						%comando per il complementare
\newcommand{\R}{\mathbb{R}}										%per scrivere l'insieme dei reali, in breve




	% -------------------------- BODY -------------------------- 

\begin{document}
	\begin{abstract}
		In questo documento daremo la dimostrazione del Teorema di Shapley vista durante la lezione del 23/05/2023.
		
		Per dimostrare il teorema serviranno diversi risultati preliminari, non tutti visti a lezione.
		
		Documento scritto e formattato in \LaTeX{} da Andrea Marino. 

		Ogni segnalazione di errori è più che bene accetta. Potete scrivermi per e-mail all'indirizzo 
		\texttt{a.marino47@studenti.unipi.it}
	\end{abstract}
	
	\section*{Introduzione}
	\begin{notazione}
		Denotiamo con $N$, o con $[n]$, l'insieme $\{1,\dots,n\}\subseteq\mathbb{N}$.
		
		Ricordiamo che il gruppo simmetrico su $n$ elementi, $S_n$, è naturalmente in bigezione con l'insieme $[n]!$ delle permutazioni degli elementi di $[n]$.
	\end{notazione}
	
	\begin{definizione}
		Introduciamo l'insieme 
		\[
			V_{[n]}=\qty\big{v\in\mathrm{Fun}\qty(\mathcal{P}(N),\R)\mid v(\emptyset)=0}.
		\]
	\end{definizione}
	
	\begin{lemma}\label{lemma:vectorspace}
		$V_{[n]}$ è uno spazio vettoriale reale di dimensione $2^n-1$.
	\end{lemma}
	\begin{proof}
		Sia $V\coloneqq\mathrm{Fun}\left(\mathcal{P}(N),\R\right)$ e consideriamo la valutazione sull'insieme vuoto: 
		\[
			\begin{array}{rccl}
				\mathit{val}_{\emptyset}\colon	& V	& \longrightarrow & \R\\
				 & v & \longmapsto & v\left(\emptyset\right)
			\end{array}	
		\]
		
		$V$ è uno spazio vettoriale su $\R$ di dimensione $\qty|\mathcal{P}(N)|=2^n$. $\mathit{val}_{\emptyset}$ è lineare e $V_{[n]}=\ker\textit{val}_{\emptyset}$, mentre ovviamente $\rank \textit{val}_{\emptyset}=1$ Si conclude per la formula delle dimensioni $\dim V = \dim\ker\textit{val}_{\emptyset}+\rank\textit{val}_{\emptyset}$.
	\end{proof}
	
	Può essere utile scendere un po' di più nel dettaglio. Gli spazi vettoriali $V$ e $\R^{2^n}$ sono isomorfi. Un isomorfismo esplicito è dato da
	\[
		\begin{array}{rccc}
			\Phi\colon & V & \longrightarrow & \R^{2^n}\\
			 & v & \longmapsto & \begin{pmatrix}
			 v(\emptyset)\\
			 v(S_1)\\
			 \vdots\\
			 v(S_{2^n-1})
			 \end{pmatrix}
		\end{array}
	\]
	Con $\mathcal{P}(N)=\qty{\emptyset,S_1,\dots,S_{2^n-1}}$.
	
	\begin{oss}
		In questo contesto, risulta ovvia la linearità di $\textit{val}_{\emptyset}$. Questa funzione infatti coincide con $\pi_1\circ\Phi$, dove $\pi_1\colon\R^{2^n}\to\R$ è la proiezione sulla prima componente.
	\end{oss}
	
	In maniera analoga, possiamo pensare ogni $v\in V_{[n]}$ (precisamente, a $\restr{v}{\mathcal{P}(N)\setminus\qty{\emptyset}}$) come a un vettore di $2^n-1$ componenti.
    \begin{definizione}\label{def:carrier_game}
        Per ogni $T\in\mathscr{D}\coloneqq\mathcal{P}(N)\setminus\{\emptyset\}$ definiamo il \emph{gioco portatore} di $T$ come $\left([n],u_T\right)$, dove $u_T(S)=\begin{dcases}1 & T\subseteq S\\ 0 & T\not\subseteq S.\end{dcases}$
    \end{definizione}
    Osserviamo che l'insieme $\qty{u_T}_{T\in\mathscr{D}}$ \emph{non} corrisponde, tramite l'isomorfismo $\Phi$\footnote{Precisamente, tramite $\pi_{\mathrm{Span}\{e_1\}^{\perp}}\circ\restr{\Phi}{V_{[n]}}$} alla base canonica di $\R^{2^n-1}$.
    \begin{lemma}\label{lemma:basis}
        $\qty{u_T}_{T\in\mathscr{D}}$ è una base di $V_{[n]}$.
    \end{lemma}
    \begin{proof}
        Siano $\qty{\lambda_{T}}_{T\in\mathscr{D}}$ degli scalari reali e consideriamo una combinazione lineare nulla $\sum_{T\in\mathscr{D}}\lambda_Tu_T=0$, ovvero $\sum_{T\in\mathscr{D}}\lambda_Tu_T(S)=0\quad\forall\,S\subseteq[n]$.

        Sia $\mathscr{C}\coloneqq\qty{T\in\mathscr{D}\mid\lambda_T\neq 0}$ e supponiamo per assurdo $\mathscr{C}\ne\emptyset$. Sia $\hat{S}\in\mathscr{C}$ minimo per inclusione, ossia $T\in\mathscr{C}, T\ne\hat{S}\implies T\not\subseteq\hat{S}$.

        Ma allora si ha
        \[
            0=\sum_{T\in\mathscr{D}}\lambda_Tu_T(\hat{S})=\sum_{T\subseteq\hat{S}}\lambda_Tu_T(\hat{S}),
        \]
        dove l'ultima uguaglianza segue dalla definizione di gioco portatore di $\hat{S}$. Dalla definizione di $\hat{S}$ ricaviamo 
        \[
            \sum_{T\subseteq\hat{S}}\lambda_Tu_T(\hat{S})=\lambda_{\hat{S}}u_{\hat{S}}(\hat{S})=\lambda_{\hat{S}},
        \]
        ma $\lambda_{\hat{S}}\ne 0$. Siamo caduti in contraddizione.

        Ne segue che $\mathscr{C}=\emptyset$; dunque $\lambda_T=0\quad\forall\,T\in\mathscr{D}$ e di conseguenza $\qty{u_T}_{T\in\mathscr{D}}$ è un sistema linearmente indipendente. Ma la sua cardinalità coincide con $\dim V_{[n]}$, quindi è una base.
    \end{proof}
	
	
	\section*{Il Teorema di Shapley}
	\begin{definizione}\label{def:proprietà}
		Una funzione $\phi\colon V_{[n]}\to\R^n$ soddisfa le proprietà di
		\begin{itemize}
			\item \textbf{Efficienza} (o \emph{razionalità collettiva}) se $\sum_{i\in[n]}\phi_i(v)=v(N)$
			\item \textbf{Simmetria} se $v(S\cup\{i\})=v(S\cup\left\{j\right\})\quad\forall\,S\subseteq[n]\setminus\qty{i,j}\implies\phi_i(v)=\phi_j(v)$
			\item \textbf{Additività} se $\phi(v+w)=\phi(v)+\phi(w)\quad\forall\,v,w\in V_{[n]}$
			\item \textbf{Giocatore nullo} se $v(S\cup\{i\})=v(S)\quad\forall\,S\subseteq[n]\implies\phi_i(v)=0$.
		\end{itemize}
	\end{definizione}
    Enunciamo, e dimostriamo, un piccolo lemma che renderà pressoché immediata la dimostrazione del teorema~\ref{teor:Shapley}.
    \begin{lemma}\label{lemma:common_structure}
        Ogni $\phi\colon V_{[n]}\to\R^n$ che soddisfa le proprietà della definizione~\ref{def:proprietà} è tale che 
        \[
            \phi_i(\lambda u_T)=\begin{dcases}
                \frac{\lambda}{T} & i\in T\\
                0 & i\not\in T
            \end{dcases}
        \]
        dove $\lambda\in\R$ e $\qty{u_T}_{T\in\mathscr{D}}$ sono le funzioni introdotte nella definizione~\ref{def:carrier_game}.
    \end{lemma}
    \begin{proof}
        Sia $T\in\mathscr{D}$. Osserviamo che ogni $i\notin T$ è un giocatore nullo, infatti per ogni $S\subseteq[n]$ si ha $T\subseteq S\implies T\subseteq S\cup\{i\}$ e $T\not\subseteq S\implies T\not\subseteq S\cup\{i\}$. Quindi $u_T(S\cup\{i\})=u_T(S)\quad\forall\,i\notin T$, e per la proprietà del giocatore nullo $i\notin T\implies\phi_i(\lambda u_T)=0$.

        Siano $i,j\in T, i\ne j$. $i$ e $j$ sono giocatori simmetrici: per ogni $S\subseteq[n]\setminus\{i,j\}$ si ha $u_T(S\cup\{i\})=0$, dato che $T\not\subseteq S\cup\{i\}$ dato che $j\in T$ ma $j\notin S\cup\{i\}$; ma $u_T(S\cup\{j\})=0$ per un motivo analogo. 
        
        Ne segue che $\phi_i(\lambda u_T)=C_T\quad\forall\,i\in T$, dove $C_T$ è un'opportuna costante che dipende da $T\in\mathscr{D}$.

        Ora,
        \[
            \qty|T|\cdot C_T = \sum_{i\in T}\phi_i(\lambda u_T)=\sum_{i=1}^n\phi_i(\lambda u_T).
        \]
        Ma per efficienza
        \[
             \sum_{i=1}^n\phi_i(\lambda u_T)=\lambda u_T(N)=\lambda
        \]
        ossia $C_T=\frac{\lambda}{\qty|T|}$.
    \end{proof}
    
	\begin{definizione}
		Sia $v\in V_{[n]}$. Il \emph{valore di Shapley} di $v$ è una funzione $\mathit{Sh}\colon V_{[n]}\to\R^{n}$, dove il valore di ciascuna componente è dato da
		\[
			\mathit{Sh}_i(v)=\frac{1}{n!}\sum_{\sigma\in S_n} v\left(P_i(\sigma)\cup\{i\}\right)-v\left(P_i(\sigma)\right)
		\]	
		dove $P_i(\sigma)\coloneqq\qty{j\in[n]\mid \sigma(j)<\sigma(i)}$.
	\end{definizione}
	
	\begin{oss}\label{oss:Shapley_alternative}
		Un'espressione alternativa per il valore di Shapley è
		\begin{equation}\label{eqn:Shapley_alternative}
			\mathit{Sh}_i(v)=\frac{1}{n!}\smashoperator[r]{\sum_{S\subseteq[n]\setminus\{i\}}}\,\qty|S|!\left(n-\qty|S|-1\right)!\left(v\left(S\cup\{i\}\right)-v(S)\right).
		\end{equation}
	\end{oss}	
	\begin{proposizione}\label{prop:Shapley_proprietà}
		Il valore di Shapley soddisfa tutte le proprietà della definizione~\ref{def:proprietà}.
	\end{proposizione}
	\begin{proof} Si tratta di una verifica diretta
		\begin{itemize}
			\item \textbf{Additività} Si veda l'esercizio~\ref{es:propr_Shapley}.
			\item \textbf{Giocatore nullo} Si veda l'esercizio~\ref{es:propr_Shapley}.
			\item \textbf{Efficienza} Per l'osservazione~\ref{oss:Shapley_alternative} si ha
				\begin{equation}\label{eqn:eff_sum}
						\sum_{i=1}^{n}\mathit{Sh}_i(v)=\frac{1}{n!}\sum_{i=1}^{n}\smashoperator[r]{\sum_{S\subseteq[n]\setminus\{i\}}}\qty|S|!(n-\qty|S|-1)!\left( v(S\cup\{i\})-v(S) \right)
				\end{equation} 
    
				Fissiamo $\hat{\imath}\in[n]$ e sia $\hat{S}\subseteq[n]\setminus\qty{\hat{\imath}}$.
				\begin{itemize}
					\item Se $\abs*{\hat{S}}<n-1$ allora $\exists\,j\ne\hat{\imath}\quad j\notin\hat{S}$. Ma allora sia $\hat{S}$ sia $\hat{S}\cup\qty{\hat{\imath}}$ compaiono 
					nella sommatoria relativa all'indice $j$ e i termini $v(\hat{S}\cup\qty{\hat{\imath}}),v(\hat{S})$ si elidono nella~\eqref{eqn:eff_sum}. 
					\item Se $\abs*{\hat{S}}=n-1$ allora $\hat{S}=[n]\setminus\qty{\hat{\imath}}$. $\hat{S}$ se ne va nella somma, basta considerare $j\ne\hat{\imath}$ e si ha
					$\hat{S}=\underbrace{[n]\setminus\qty{\hat{\imath},j}}_{\in\,\mathcal{P}\left(N\setminus\qty{j}\right)}\cup\qty{j}$. 

					$\hat{S}\cup\qty{\hat{\imath}}=[n]$ non viene eliminato da nessuno, e compare in tutto $n$ volte nella sommatoria più esterna
					(una per ciascun indice $i\in[n]$). Quindi nella~\eqref{eqn:eff_sum} rimane $\frac{1}{n!}\cdot (n-1)!\,n\cdot v(N)=v(N)$.
				\end{itemize}
			\item \textbf{Simmetria} Siano $\hat{\imath},\hat{\jmath}\in[n], \hat{\imath}\ne \hat{\jmath}$ giocatori simmetrici. Possiamo riscrivere la~\eqref{eqn:Shapley_alternative} come:
			\begin{flalign}
				\mathit{Sh}_i(v) &= 
				\begin{multlined}[t]%\SwapAboveDisplaySkip
					\frac{1}{n!} \smashoperator[r]{\sum_{\crampedsubstack{S\subseteq[n]\setminus\{\hat{\imath}\}\\\hat{\jmath}\in S}}}\qty|S|!(n-\qty|S|-1)!\left( v(S\cup\qty{\hat{\imath}})-v(S) \right)+\\
					\shoveright{-\frac{1}{n!}\smashoperator[r]{\sum_{S\subseteq[n]\setminus\{\hat{\imath},\hat{\jmath}\}}}\qty|S|!(n-\qty|S|-1)!\left( v(S\cup\qty{\hat{\imath}})-v(S) \right)=}\\
				\end{multlined} & \nonumber\\
				& \begin{aligned}\label{eqn:sum_twoterms}
					=\frac{1}{n!} \smashoperator[r]{\sum_{\crampedsubstack{S\subseteq[n]\setminus\{\hat{\imath}\}\\\hat{\jmath}\in S}}}\qty|S|!(n-\qty|S|-1)!\left( v(S\cup\qty{\hat{\imath}})-v(S) \right)+\\
					-\frac{1}{n!}\smashoperator[r]{\sum_{S\subseteq[n]\setminus\{\hat{\imath},\hat{\jmath}\}}}\qty|S|!(n-\qty|S|-1)!\left( v(S\cup\qty{\hat{\jmath}})-v(S) \right)
				\end{aligned}	& 		
			\end{flalign}

            Per ogni $S\subseteq[n]\setminus\{\hat{\imath}\}$ definiamo %$S^{-}\coloneqq S\setminus\{\hat{\jmath}\}$, $S^{+}\coloneqq S^-\setminus\{\hat{\imath}\}=\left(S\cup\{\hat{\imath}\}\right)\setminus\{\hat{\jmath}\}$ 
            %\begin{subequations}
            \begin{align*}
                S^{-} &\coloneqq S\setminus\{\hat{\jmath}\} \\
                S^{+} &\coloneqq S^-\setminus\{\hat{\imath}\}=\left(S\cup\{\hat{\imath}\}\right)\setminus\{\hat{\jmath}\}.
            \end{align*}
            %\end{subequations}
            Osserviamo che 
            \begin{gather}
                S\cup\{\hat{\imath}\}=S^-\cup\{\hat{\imath},\hat{\jmath}\}=S^+\cup\{\hat{\jmath}\}\nonumber\\
                \intertext{e che dunque}
                \begin{split}\label{eqn:v_S}
                    v(S\cup\{\hat{\imath}\}-v(S) &= v(S^+\cup\{\hat{j}\})-v(S^-\cup\{\hat{\jmath}\})\\
                    &=v(S^-\cup\{\hat{\imath},\hat{\jmath}\})-v(S^-\cup\{\hat{\jmath}\}).
                \end{split}
            \end{gather}
            Dalla~\eqref{eqn:v_S} segue che
            \begin{align}
                & \frac{1}{n!} \smashoperator[r]{\sum_{\crampedsubstack{S\subseteq[n]\setminus\{\hat{\imath}\}\\\hat{\jmath}\in S}}}\qty|S|!(n-\qty|S|-1)!\left( v(S\cup\qty{\hat{\imath}})-v(S) \right)=\nonumber\\
                & \!\begin{multlined}[b]
                   = \frac{1}{n!}\smashoperator[r]{\sum_{S^-\subseteq[n]\setminus\{\hat{\imath},\hat{\jmath}\}}}\qty|S^-\cup\{\hat{\jmath}\}|!\left(n-\qty|S^-\cup\{\hat{\jmath}\}|-1\right)!\\
                    \left(v(S^-\cup\{\hat{\imath},\hat{\jmath}\})-v(S^-\cup\{\hat{\jmath}\})\right)
                \end{multlined}= \nonumber\\
                %\intertext{I giocatori $\hat{\imath}$ e $\hat{\jmath}$ sono simmetrici e non appartengono a $S^-$. Dunque la precedente è uguale a}
                & \!\begin{multlined}[b]\label{eqn:sum_S-}
                    = \frac{1}{n!}\smashoperator[r]{\sum_{S^-\subseteq[n]\setminus\{\hat{\imath},\hat{\jmath}\}}}\qty|S^-\cup\{\hat{\jmath}\}|!\left(n-\qty|S^-\cup\{\hat{\jmath}\}|-1\right)!\\
                    \left(v(S^-\cup\{\hat{\imath},\hat{\jmath}\})-v(S^-\cup\{\hat{\imath}\})\right),
                \end{multlined}
            \end{align}
            dove l'ultima uguaglianza segue dal fatto che i giocatori $\hat{\imath}$ e $\hat{\jmath}$ sono simmetrici e non appartengono a $S^-$. Ora possiamo riscrivere la~\eqref{eqn:sum_S-} nella forma
            \[
                \frac{1}{n!} \smashoperator[r]{\sum_{\crampedsubstack{S^+\subseteq[n]\setminus\{\hat{\jmath}\}\\\hat{\imath}\in S^+}}}\qty|S^+|!(n-\qty|S^+|-1)!\left( v(S^+\cup\qty{\hat{\jmath}})-v(S^+) \right).
            \]
            Sostituendo nella~\eqref{eqn:sum_twoterms} otteniamo
            \begin{flalign*}
                \mathit{Sh}_i(v) =& \frac{1}{n!} \smashoperator[r]{\sum_{\crampedsubstack{S\subseteq[n]\setminus\{\hat{\jmath}\}\\\hat{\imath}\in S}}}\qty|S|!(n-\qty|S|-1)!\left( v(S\cup\qty{\hat{\jmath}})-v(S) \right)+&\\
                 & +\frac{1}{n!}\smashoperator[r]{\sum_{S\subseteq[n]\setminus\{\hat{\imath},\hat{\jmath}\}}}\qty|S|!(n-\qty|S|-1)!\left( v(S\cup\qty{\hat{\jmath}})-v(S) \right)=&\\
                 &  =\mathit{Sh}_j(v). 
            \end{flalign*}
            Dunque la funzione $\mathit{Sh}$ è simmetrica rispetto ai giocatori $\hat{\imath}$ e $\hat{\jmath}$.\qedhere
		\end{itemize}
	\end{proof}


	% Esperimenti
%	\[
%		\mathit{Sh}_i(v) =
%		\begin{multlined}[t]%[\columnwidth]
%			\frac{1}{n!} \smashoperator[r]{\sum_{\crampedsubstack{S\subseteq[n]\setminus\{\hat{\imath}\}\\\hat{\jmath}\in S}}}\qty|S|!(n-\qty|S|-1)!\left( v(S\cup\qty{\hat{\imath}})-v(S) \right)+ \\
%			-\frac{1}{n!}\smashoperator[r]{\sum_{S\subseteq[n]\setminus\{\hat{\imath},\hat{\jmath}\}}}\qty|S|!(n-\qty|S|-1)!\left( v(S\cup\qty{\hat{\imath}})-v(S) \right)=
%		\end{multlined}
%	\]

%	\begin{flalign*}
%		\mathit{Sh}_i(v) &=
%		\begin{multlined}[t]
%			\frac{1}{n!} \smashoperator[r]{\sum_{\crampedsubstack{S\subseteq[n]\setminus\{\hat{\imath}\}\\\hat{\jmath}\in S}}}\qty|S|!(n-\qty|S|-1)!\left( v(S\cup\qty{\hat{\imath}})-v(S) \right)+\\
%			\shoveright{\frac{1}{n!}-\smashoperator[lr]{\sum_{S\subseteq[n]\setminus\{\hat{\imath},\hat{\jmath}\}}}\qty|S|!(n-\qty|S|-1)!\left( v(S\cup\qty{\hat{\imath}})-v(S) \right)= }
%		\end{multlined} & \\
%		& \begin{multlined}[t]
%			\frac{1}{n!} \smashoperator[r]{\sum_{\crampedsubstack{S\subseteq[n]\setminus\{\hat{\imath}\}\\\hat{\jmath}\in S}}}\qty|S|!(n-\qty|S|-1)!\left( v(S\cup\qty{\hat{\imath}})-v(S) \right)+ \\
%			\frac{1}{n!}-\smashoperator[lr]{\sum_{S\subseteq[n]\setminus\{\hat{\imath},\hat{\jmath}\}}}\qty|S|!(n-\qty|S|-1)!\left( v(S\cup\qty{\hat{\jmath}})-v(S) \right)= 
%		\end{multlined} &
%	\end{flalign*}

	
	\begin{teorema}\label{teor:Shapley}
		Il valore di Shapley è l'unica funzione che soddisfa le proprietà di efficienza, simmetria, additività e la proprietà del giocatore nullo. 
	\end{teorema}
	\begin{proof}
        Per il lemma~\ref{lemma:vectorspace} $V_{[n]}$ è uno spazio vettoriale su $\R$ di dimensione $2^n-1$, una cui base è data -- per il lemma~\ref{lemma:basis} -- da $\qty{u_T}_{T\in\mathscr{D}}$.

        Sia $\phi\colon V_{[n]}\to\R^n$ una funzione che soddisfa le quattro proprietà della definizione~\ref{def:proprietà}. Per la proposizione~\ref{prop:Shapley_proprietà} anche la funzione $\mathit{Sh}$ soddisfa tali proprietà, ma quindi per il lemma~\ref{lemma:common_structure} si ha che
        \begin{equation}\label{eqn:Sh_phi_i}
            \phi_i(\lambda u_T)=\mathit{Sh}_i(\lambda u_T)\quad\forall\,i\in[n].
        \end{equation}
        Sia $v\in V_{[n]}$. Poiché $\qty{u_T}_{T\in\mathscr{D}}$ è una base, esistono degli scalari $\qty{\lambda_T}_{T\in\mathscr{D}}$ tali che $v=\sum_{T\in\mathscr{D}}\lambda_T u_T$. Ora, per ogni $i\in[n]$ si ha
        \[
            \phi_i(v) = \phi_i\qty(\sum_{T\in\mathscr{D}}\lambda_Tu_T) = \sum_{T\in\mathscr{D}}\phi_i(\lambda_Tu_T)        
        \]
        poiché $\phi$ soddisfa la proprietà di additività. Ma per la~\eqref{eqn:Sh_phi_i}
        \begin{align*}
            \sum_{T\in\mathscr{D}}\phi_i(\lambda_Tu_T)&=\sum_{T\in\mathscr{D}}\mathit{Sh}_i(\lambda_Tu_T)=\\
             &=\mathit{Sh}_i\qty(\sum_{T\in\mathscr{D}}\lambda_Tu_T)=\mathit{Sh}_i(v).\qedhere
        \end{align*}
	\end{proof}


	\section*{Esercizi}
	\begin{esercizio}
		Dimostrare direttamente che $\mathrm{Fun}\qty(\mathcal{P}(N),\R)$ è uno spazio vettoriale reale.
	\end{esercizio}
	\begin{esercizio}
		Dimostrare il contenuto dell'osservazione~\ref{oss:Shapley_alternative}.
	\end{esercizio}
	\begin{esercizio}\label{es:propr_Shapley}
		Dimostrare che il valore di Shapley soddisfa la proprietà di additività e la proprietà del giocatore nullo.
	\end{esercizio} 
    \begin{esercizio}
        Dimostrare che nell'insieme $\mathscr{C}$ introdotto nella dimostrazione del teorema~\ref{teor:Shapley}, nell'ipotesi (assurda) in cui esso sia non vuoto, un minimo per inclusione $\hat{S}$ esiste. 
    \end{esercizio}

\end{document}
