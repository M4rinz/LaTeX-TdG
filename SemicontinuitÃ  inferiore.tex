\documentclass[a4paper, 11pt]{article}
\usepackage[T1]{fontenc}
\usepackage[utf8]{inputenc}
\usepackage[english,italian]{babel}

\usepackage{amssymb}
%\usepackage{amsmath}
\usepackage{mathtools}
\usepackage{amsthm}

\usepackage{microtype}

\usepackage{hyperref}

\title{Semicontinuità Inferiore}
\author{Andrea Marino}
\date{Maggio 2023}

% comandi custom:
\newcommand{\R}{\mathbb{R}}
\newcommand{\I}[1]{\mathcal{I}(#1)}

% teoremi custom:
\theoremstyle{plain} 	% Definisce ambienti per Teoremi, esercizi, corollari... Con lo stile adeguato
	\newtheorem{proposizione}{Proposizione}
	\newtheorem*{proposizione*}{Proposizione}

    \newtheorem{lemma}{Lemma}
	
	\newtheorem{teorema}{Teorema}
	\newtheorem*{teorema*}{Teorema}

    \newtheorem{corollario}{Corollario}

\theoremstyle{definition}
    \newtheorem{definizione}{Definizione}
	\newtheorem*{definizione*}{Definizione}

    \newtheorem{es}{Esercizio}

\theoremstyle{remark}
	\newtheorem{oss}{Osservazione}
	\newtheorem*{oss*}{Osservazione}

\begin{document}

    \maketitle
    
    \begin{abstract}
        Questo breve articolo esamina tre caratterizzazioni equivalenti di semicontinuità inferiore. 

        I concetti trattati sono alla portata di chiunque abbia qualche nozione di base di topologia generale (all'Università di Pisa, queste sono fornite ad esempio dal corso di Geometria 2), ma può comunque far comodo 
        averli in un documento a parte, dal momento che se n'è fatto uso durante il corso.
        \smallskip

        Ogni segnalazione di errori è più che bene accetta. Potete scrivermi per e-mail all'indirizzo 
		\texttt{a.marino47@studenti.unipi.it}
    \end{abstract}

    Sia $\left(X,\tau\right)$ uno spazio topologico. Per il resto della trattazione ometteremo di indicare esplicitamente la topologia $\tau$ di $X$, visto che non ci saranno ambiguità.
    \begin{definizione}
        Una funzione $f\colon X\to\R$ si dice \emph{semicontinua inferiormente} in $x_0\in X$ se $\forall\,\epsilon>0\ \exists\,U\in\I{x_0}\quad f(x)>f(x_0)-\epsilon$, dove $\I{x_0}$ è la famiglia degli intorni di $x_0$.
    \end{definizione}
    La topologia considerata su $\R$ è, naturalmente, la topologia euclidea.
    \begin{oss*}
        Una funzione si dice \emph{semicontinua inferiormente su $X$} se è semicontinua inferiormente in ogni $x\in X$.
    \end{oss*}
    \begin{definizione}\label{def:top_sci}
        La \emph{topologia della semicontinuità inferiore} su $\R$ è data da $\tau_I=\left\{\emptyset,\R\right\}\cup\left\{(y,+\infty)\mid y\in\R\right\}$.
    \end{definizione}
    Quindi gli aperti sono $\emptyset, \R$ e le semirette aperte a sinistra, illimitate a destra.

    Ricordiamo che una funzione tra spazi topologici si dice \emph{continua} se l'immagine inversa di un aperto è un aperto. Ricordiamo inoltre la seguente definizione:

    \begin{definizione}\label{def:cont_pt}
        Sia $A$ uno spazio topologico. Una funzione $f\colon X\to A$ si dice \emph{continua in $x_0$} se $\forall\,V\in\I{f(x_0)}\ \exists\,U\in\I{x_0}\quad f^{-1}\left(V\right)\subseteq U$.
    \end{definizione}

    \begin{proposizione}\label{prop:con_top}
        $f\colon X\to\R$ è semicontinua inferiormente se e solo se $f\colon X\to\left(\R,\tau_I\right)$ è continua
    \end{proposizione}
    \begin{proof} Ricordiamo che una funzione è continua se e solo se è continua in ogni punto.
        \begin{description}
            \item[$\left(\Longrightarrow\right)$] Consideriamo $x_0\in X$ e sia $V\in\I{f(x_0)}$. Possiamo supporre $V=(f(x_0)-\epsilon,+\infty)$ per un qualche $\epsilon>0$. Ora, per ipotesi $\exists\,U\in\I{x_0}$ tale che $f(x)>f(x_0)-\epsilon\quad\forall\,x\in U$, ovvero $f(x)\in V\quad\forall\,x\in U$. Ma quindi $f\left(U\right)\subseteq V$. 

            Per la definizione~\ref{def:cont_pt} $f$ è continua in $x_0$. Dall'arbitrarietà di $x_0$ segue che $f$ è continua.
            \item[$\left(\Longleftarrow\right)$] Sia $x_0\in X$ e sia $\epsilon>0$. \`E sufficiente dimostrare che $f$ è semicontinua inferiormente in $x_0$.
            
            Consideriamo $f(x_0)\in\R$ e l'intorno $V\in\I{f(x_0)}$ definito da $V=(f(x_0)-\epsilon,+\infty)$. Poiché $f$ è continua, $U\coloneqq f^{-1}\left(V\right)$ è un aperto di $X$. Si ha che $x_0\in U$, infatti $x_0\in f^{-1}(f(x_0))$, quindi $U$ è un aperto che contiene $x_0$, ossia $U\in\I{x_0}$. 

            Ora, $f\left(U\right)=f\left(f^{-1}(V)\right)\subseteq V$ ossia $\forall\,x\in U$ si ha $f(x)\in V$. Dunque $f(x)>f(x_0)-\epsilon\quad\forall\,x\in U$. \qedhere
        \end{description}
    \end{proof}

    \`E possibile dare un'altra caratterizzazione della semicontinuità inferiore. Prima però ricordiamo la seguente definizione:
    \begin{definizione} il \emph{limite inferiore di $f$ in $x_0$} è definito come
        \begin{align*}
            & \liminf_{x\to x_0}f(x)\coloneqq\sup_{U\in\I{x_0}}\inf_{x\in U\setminus\{x_0\}} f(x)=\\
            & =\sup\left\{\inf\left\{f(x)\in\R\mid x\in U\setminus\{x_0\}\right\}\mid U\in\I{x_0}\right\}.            
        \end{align*}
    \end{definizione}
    \begin{proposizione}\label{prop:liminf}
        Sia $x_0\in X$ un punto di accumulazione. $f\colon X\to\R$ è semicontinua inferiormente in $x_0\in X$ se e solo se $\liminf_{x\to x_0}f(x)\ge f(x_0)$
    \end{proposizione}
    \begin{proof}
        \begin{description}
            \item[$\left(\Longleftarrow\right)$] 
            Sia $\epsilon>0$ e per ogni $U\in\I{x_0}$ definiamo $i(U)=\inf_{x\in U\setminus\{x_0\}}$.

            Per ipotesi 
            \begin{equation}\label{eqn:sup_ineq}
                \liminf_{x\to x_0}=\sup\left\{i(U)\mid U\in\I{x_0}\right\}\ge f(x_0)>f(x_0)-\epsilon.
            \end{equation}
            Se per ogni $U\in\I{x_0}$ fosse $f(x)\le f(x_0)-\epsilon\quad\forall\,x\in U\setminus\{x_0\}$ allora si avrebbe che $i(U)\le f(x_0)-\epsilon\quad\forall\,U\in\,\I{x_0}$, e di conseguenza -- passando all'estremo superiore in $U\in\I{x_0}$ -- si avrebbe una contraddizione con la~\eqref{eqn:sup_ineq}.

            Dunque deve esistere $\I{x_0}\ni U=U(\epsilon)$ tale che $f(x)>f(x_0)-\epsilon\quad\forall\,x\in U\setminus\{x_0\}$.
            \item[$\left(\Longrightarrow\right)$] Per contronominale. Usando la notazione introdotta al punto precedente, $\liminf_{x\to x_0}f(x)<f(x_0)\iff\sup_{U\in\I{x_0}}i(U)<f(x_0)$. Di conseguenza $\exists\,\epsilon>0\quad\sup_{U\in\I{x_0}}i(U)\le f(x_0)-\epsilon$, ma quindi $i(U)\le f(x_0)-\epsilon\quad\forall\,U\in\I{x_0}$.

            Cioè $\inf_{x\in U\setminus\{x_0\}}f(x)\le f(x_0)-\epsilon$. Da ciò segue che $f(x)\le f(x_0)-\epsilon\quad\forall\,x\in U\setminus\{x_0\}\,\forall\,U\in\I{x_0}$.\qedhere
        \end{description}
    \end{proof}

    Supponiamo ora che $X=\R^{n}$. In questo caso, possiamo dare un'ulteriore caratterizzazione di semicontinuità inferiore.
    \begin{proposizione}
        $f\colon\R^{n}\to\R$ è semicontinua inferiormente se e solo se tutti i sottolivelli sono chiusi.
    \end{proposizione}
    \begin{proof}
        Per la proposizione~\ref{prop:con_top} $f$ è semicontinua inferiormente se e solo se $f\colon\R^n\to\left(\R,\tau_I\right)$ è continua. Ciò accade se e solo se l'immagine inversa di un chiuso è un chiuso.

        Osserviamo che, per ogni $\alpha\in\R$, $(-\infty,\alpha]=\R\setminus\left(\alpha,+\infty\right)$ è chiuso in $\tau_I$ in quanto complementare di un aperto. Ma $f^{-1}\left(\left(-\infty,\alpha\right]\right)=\left\{x\in\R^n\mid f(x)\le\alpha\right\}$, ossia l'immagine inversa di $(-\infty,\alpha]$ è proprio il sottolivello di livello $\alpha$.
    \end{proof}

    Vediamo adesso una proprietà molto utile delle funzioni semicontinue inferiormente. Prima di fare ciò, introduciamo un semplice lemma. La proprietà segue immediatamente da questo.
    \begin{lemma}\label{lemma:cpt_inf}
        Sia $K\subseteq\R$ compatto per la topologia della semicontinuità inferiore. Allora $\inf K\in K$, ossia $K$ ammette minimo.        
    \end{lemma}
    \begin{proof}
        Sia $\alpha\coloneqq\inf K$ e supponiamo per assurdo che $\alpha\notin K$. 

        Osserviamo che $\mathcal{U}=\left\{\left(\alpha+\frac{1}{n},+\infty\right)\right\}_{n>0}$ è un ricoprimento aperto di $K$. Sia infatti $x\in K$. Allora $x>\alpha$, quindi $\exists\,\Tilde{n}\in\mathbb{N}\setminus\{0\}\quad \frac{1}{\Tilde{n}}<x-\alpha$, dunque $x\in\left(\alpha+1/\Tilde{n},+\infty\right)$.

        Ora, 
        \[
            \bigcup\mathcal{U}=\bigcup_{n=1}^{+\infty}(\alpha+1/n,+\infty)=\left(\alpha,+\infty\right)\supseteq K,
        \]
        ma non possono esserci sottoricoprimenti finiti di $\mathcal{U}$.

        Siano infatti $n_0<n_1<\dots<n_k$ degli indici. Si ha 
        \[
            \bigcup_{t=1}^k(\alpha+1/n_t,+\infty)=\left(\alpha+\frac{1}{n_k},+\infty\right)
        \]
        che non contiene $K$ poiché $\alpha$ è l'estremo inferiore di $K$. Siamo caduti in contraddizione.
    \end{proof}
    \begin{corollario}
        Sia $X$ compatto e $f\colon X\to\R$ una funzione semicontinua inferiormente. Allora $f$ assume minimo su $X$.
    \end{corollario}
    \begin{proof}
        Per la proposizione~\ref{prop:con_top} $f$ è continua se consideriamo $\R$ con la topologia della semicontinuità inferiore. 

        Sappiamo dalla topologia generale che l'immagine di un compatto tramite una funzione continua è un compatto, quindi $f(X)$ è compatto in $\left(\R,\tau_I\right)$. Per il lemma~\ref{lemma:cpt_inf} $f(X)$ ammette minimo.
    \end{proof}
    
    

    


    \section*{Esercizi}
    \begin{es}
        Verificare che la topologia della semicontinuità inferiore introdotta nella definizione~\ref{def:top_sci} è effettivamente una topologia su $\mathbb{R}$ 
    \end{es}
    \begin{es}
        Dimostrare che una funzione è continua se e solo se è continua in ogni punto
    \end{es}
    \begin{es}
        Perché nella dimostrazione della proposizione~\ref{prop:con_top} abbiamo potuto supporre $V=(f(x_0)-\epsilon,+\infty)$ per un qualche $\epsilon>0$?  
    \end{es}
    \begin{es}
        Cosa ci assicura che la quantità $i(U)$ (introdotta nella dimostrazione della proposizione~\ref{prop:liminf}) è ben definita? 
    \end{es}
    \begin{es}
        Sia $x\in\R$. Dimostrare che $[x,+\infty)$ è compatto in $\left(\R,\tau_I\right)$
    \end{es}
    \begin{es}
        Sia $K\subseteq\left(\R,\tau_I\right)$. Dimostrare che la condizione $\min K\in K$ è sufficiente alla compattezza
    \end{es}
    \begin{es}
        Dimostrare che $\bigcup_{n=1}^{+\infty}(\alpha+1/n,+\infty)=\left(\alpha,+\infty\right)$
    \end{es}
    \begin{es}
        Descrivere la topologia della semicontinuità superiore e, partendo da essa, dimostrare le caratterizzazioni delle funzioni semicontinue superiormente analoghe a quelle viste
    \end{es}
\end{document}
